\chapter{Introduction}

\section{Description}
\par
In brief \gnutls{} can be described as a portable library which offers
an API to access secure communication protocols. These protocols provide
privacy over insecure lines, and were designed to prevent 
eavesdropping, tampering, or message forgery.

\par
Technically \gnutls{} is a library which implements the \tlsI{} and 
\sslIII{} protocols.
\tls{} stands for 'Transport Layer Security' and is the sucessor of \ssl{}, 
the Secure Sockets Layer protocol designed by Netscape. 

\tlsI{}\footnote{described in {\it RFC 2246}} is an Internet protocol,
defined by {IETF}\footnote{IETF or Internet Engineering Task Force 
is a large open international community of network
designers, operators, vendors, and researchers concerned with the evolution of 
the Internet architecture and the smooth operation of the Internet. It is open to any interested individual.}, 
that provides confidentiality, and authentication layers over any reliable
transport layer.
The above protocols are implemented in a reentrant way. 
This allows multiple threads of execution, without the need for critical 
sections and locks. 

\par
See \htmladdnormallink{http://www.gnutls.org/}{http://www.gnutls.org/}
and \htmladdnormallink{http://www.gnu.org/software/gnutls/}{http://www.gnu.org/software/gnutls/} 
for updated versions of the \gnutls{} software and this document.

\section{Current state}

Currently \gnutls{} implements:
\begin{itemize}
\item the \tlsI{} and \sslIII{} protocols.
\item {\bf X.509} Public Key Infrastructure.
\item {\bf OpenPGP} Public Key Infrastructure.
\item {\bf SRP} for \tls{} authentication.
\item \tls{} {\bf Extension mechanism}.
\end{itemize}

