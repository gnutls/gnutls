\documentclass{article}
\begin{document}

\oddsidemargin=0pt
\marginparsep = 0pt
\marginparwidth=0pt
\textwidth=6cm

\title{Transport Layer Security}
\author{Nikos Mavroyanopoulos}
\maketitle

\section{Introduction}
\par
gnuTLS is a library which implements the {\bf TLS 1.0} and {\bf SSL 3.0} protocols.
TLS stands for 'Transport Layer Security' and is the sucessor of SSL (Secure Sockets Layer).
{\bf TLS 1.0} is described is {\it RFC 2246} and is an Internet protocol (thus it's mostly used over TCP/IP),
that provides confidentiality, and authentication layers. 

\subsection{Confidentiality}
\par
Confidentiality is provided by using symmetric encryption algorithms like {\bf 3DES}, {\bf AES}, or
stream algorithms like {\bf ARCFOUR}. A symmetric encryption algorithm uses a single (secret) key
to encrypt and decrypt data.

\subsection{Authentication}
\par
The following authentication schemas are supported in gnuTLS:
\begin{enumerate}
 \item X509 Public Key Infrastructure
 \item Anonymous authentication
 \item SRP authentication
\end{enumerate}



\section{X509 Client Examples}

Let's assume now that we want to create a client which communicates
with servers using the X509 authentication schema. The following client
is a very simple TLS client, it does not support session resuming nor
any other fancy features.

\subsection{Simple Client example}
\begin{verbatim}

#include <stdio.h>
#include <stdlib.h>
#include <sys/types.h>
#include <sys/socket.h>
#include <netinet/in.h>
#include <arpa/inet.h>
#include <gnutls.h>

#define MAX_BUF 1024
#define CRLFILE "crl.pem"
#define CAFILE "ca.pem"
#define SA struct sockaddr
#define MSG "GET / HTTP/1.0\r\n\r\n"

int main()
{
   const char *PORT = "443";
   const char *SERVER = "127.0.0.1";
   int err, ret;
   int sd, ii;
   struct sockaddr_in sa;
   GNUTLS_STATE state;
   char buffer[MAX_BUF + 1];
   X509PKI_CLIENT_CREDENTIALS xcred;

   if (gnutls_global_init() < 0) {
      fprintf(stderr, "global state initialization error\n");
      exit(1);
   }
   /* X509 stuff */
   if (gnutls_allocate_x509_client_sc(&xcred, 0) < 0) {  /* no client private key */
      fprintf(stderr, "memory error\n");
      exit(1);
   }
   /* set's the trusted cas file
    */
   gnutls_set_x509_client_trust(xcred, CAFILE, CRLFILE);

   /* connects to server 
    */
   sd = socket(AF_INET, SOCK_STREAM, 0);

   memset(&sa, '\0', sizeof(sa));
   sa.sin_family = AF_INET;
   sa.sin_port = htons(atoi(PORT));
   inet_pton(AF_INET, SERVER, &sa.sin_addr);

   err = connect(sd, (SA *) & sa, sizeof(sa));
   if (err < 0) {
      fprintf(stderr, "Connect error\n");
      exit(1);
   }
   /* Initialize TLS state 
    */
   gnutls_init(&state, GNUTLS_CLIENT);

   /* allow both SSL3 and TLS1
    */
   gnutls_set_protocol_priority(state, GNUTLS_TLS1, GNUTLS_SSL3, 0);

   /* allow only ARCFOUR and 3DES ciphers
    * (3DES has the highest priority)
    */
   gnutls_set_cipher_priority(state, GNUTLS_3DES_CBC, GNUTLS_ARCFOUR, 0);

   /* only allow null compression
    */
   gnutls_set_compression_priority(state, GNUTLS_NULL_COMPRESSION, 0);

   /* use GNUTLS_KX_RSA
    */
   gnutls_set_kx_priority(state, GNUTLS_KX_RSA, 0);

   /* allow the usage of both SHA and MD5
    */
   gnutls_set_mac_priority(state, GNUTLS_MAC_SHA, GNUTLS_MAC_MD5, 0);


   /* put the x509 credentials to the current state
    */
   gnutls_set_cred(state, GNUTLS_X509PKI, xcred);


   /* Perform the TLS handshake
    */
   ret = gnutls_handshake(sd, state);

   if (ret < 0) {
      fprintf(stderr, "*** Handshake failed\n");
      gnutls_perror(ret);
      goto end;
   } else {
      printf("- Handshake was completed\n");
   }

   gnutls_write(sd, state, MSG, strlen(MSG));

   ret = gnutls_read(sd, state, buffer, MAX_BUF);
   if (gnutls_is_fatal_error(ret) == 1 || ret == 0) {
      if (ret == 0) {
         printf("- Peer has closed the GNUTLS connection\n");
         goto end;
      } else {
         fprintf(stderr, "*** Received corrupted data(%d) - server has terminated the connection abnormally\n",
                 ret);
         goto end;
      }
   } else {
      if (ret == GNUTLS_E_WARNING_ALERT_RECEIVED || ret == GNUTLS_E_FATAL_ALERT_RECEIVED)
         printf("* Received alert [%d]\n", gnutls_get_last_alert(state));
      if (ret == GNUTLS_E_REHANDSHAKE)
         printf("* Received HelloRequest message (server asked to rehandshake)\n");
   }

   if (ret > 0) {
      printf("- Received %d bytes: ", ret);
      for (ii = 0; ii < ret; ii++) {
         fputc(buffer[ii], stdout);
      }
      fputs("\n", stdout);
   }
   gnutls_bye(sd, state, GNUTLS_SHUT_RDWR);

 end:

   shutdown(sd, SHUT_RDWR);     /* no more receptions */
   close(sd);

   gnutls_deinit(state);

   gnutls_free_x509_client_sc(xcred);

   gnutls_global_deinit();

   return 0;
}

\end{verbatim}


\subsection{Getting peer's information}
\par The above example was the simplest form of a client, it didn't even check
the result of the peer's certificate verification function (ie. if we have
an authenticated connection. The following function does check the peer's X509
Certificate, and prints some information about the current state.
\par
This function should be called after a successful \begin{verbatim}gnutls_handshake()\end{verbatim}.

\begin{verbatim}

#define PRINTX(x,y) if (y[0]!=0) printf(" -   %s %s\n", x, y)
#define PRINT_DN(X) PRINTX( "CN:", x509_info->X.common_name); \
	PRINTX( "OU:", x509_info->X.organizational_unit_name); \
	PRINTX( "O:", x509_info->X.organization); \
	PRINTX( "L:", x509_info->X.locality_name); \
	PRINTX( "S:", x509_info->X.state_or_province_name); \
	PRINTX( "C:", x509_info->X.country);

int print_info(GNUTLS_STATE state)
{
   const char *tmp;
   const X509PKI_CLIENT_AUTH_INFO *x509_info;

   /* print the key exchange's algorithm name
    */
   tmp = gnutls_kx_get_name(gnutls_get_current_kx(state));
   printf("- Key Exchange: %s\n", tmp);

   /* in case of X509 PKI
    */
   if (gnutls_get_auth_info_type(state) == GNUTLS_X509PKI) {
      x509_info = gnutls_get_auth_info(state);
      if (x509_info != NULL) {
         switch (x509_info->peer_certificate_status) {
         case GNUTLS_CERT_NOT_TRUSTED:
            printf("- Peer's X509 Certificate was NOT verified\n");
            break;
         case GNUTLS_CERT_EXPIRED:
            printf("- Peer's X509 Certificate was verified but is expired\n");
            break;
         case GNUTLS_CERT_TRUSTED:
            printf("- Peer's X509 Certificate was verified\n");
            break;
         case GNUTLS_CERT_INVALID:
         default:
            printf("- Peer's X509 Certificate was invalid\n");
            break;

         }
      }
   }
   printf(" - Certificate info:\n");
   printf(" - Certificate version: #%d\n", x509_info->peer_certificate_version);

   PRINT_DN(peer_dn);

   printf(" - Certificate Issuer's info:\n");
   PRINT_DN(issuer_dn);


   tmp = gnutls_version_get_name(gnutls_get_current_version(state));
   printf("- Version: %s\n", tmp);

   tmp = gnutls_compression_get_name(gnutls_get_current_compression_method(state));
   printf("- Compression: %s\n", tmp);

   tmp = gnutls_cipher_get_name(gnutls_get_current_cipher(state));
   printf("- Cipher: %s\n", tmp);

   tmp = gnutls_mac_get_name(gnutls_get_current_mac_algorithm(state));
   printf("- MAC: %s\n", tmp);

   return 0;
}

\end{verbatim}


\subsection{Resuming Sessions}
\par
The \begin{verbatim}gnutls_handshake()\end{verbatim} function, is expensive since
a lot of calculations are performed. In order to support many fast connections to
the same server a client may use session resuming. {\bf Session resuming} is a
feature of the {\bf TLS} protocol which allows a client to connect to a server,
after a successful handshake, without the expensive calculations (ie. use the previously
established keys). {\bf gnuTLS} supports this feature, and this example illustrates a
typical use of it. (This is a modification of the simple client example)

\par
Keep in mind that sessions are expired after some time (for security reasons), thus
it may be normal for a server not to resume a session even if you requested that.

\begin{verbatim}

#include <stdio.h>
#include <stdlib.h>
#include <sys/types.h>
#include <sys/socket.h>
#include <netinet/in.h>
#include <arpa/inet.h>
#include <gnutls.h>

#define MAX_BUF 1024
#define CRLFILE "crl.pem"
#define CAFILE "ca.pem"
#define SA struct sockaddr
#define MSG "GET / HTTP/1.0\r\n\r\n"

int main()
{
   const char *PORT = "443";
   const char *SERVER = "127.0.0.1";
   int err, ret;
   int sd, ii;
   struct sockaddr_in sa;
   GNUTLS_STATE state;
   char buffer[MAX_BUF + 1];
   X509PKI_CLIENT_CREDENTIALS xcred;
   /* variables used in session resuming */
   int t;
   char *session;
   char *session_id;
   int session_size;
   int session_id_size;
   char *tmp_session_id;
   int tmp_session_id_size;

   if (gnutls_global_init() < 0) {
      fprintf(stderr, "global state initialization error\n");
      exit(1);
   }
   /* X509 stuff */
   if (gnutls_allocate_x509_client_sc(&xcred, 0) < 0) {  /* no client private key */
      fprintf(stderr, "memory error\n");
      exit(1);
   }
   gnutls_set_x509_client_trust(xcred, CAFILE, CRLFILE);

   for (t = 0; t < 2; t++) {    /* connect 2 times to the server */

      sd = socket(AF_INET, SOCK_STREAM, 0);
      memset(&sa, '\0', sizeof(sa));
      sa.sin_family = AF_INET;
      sa.sin_port = htons(atoi(PORT));
      inet_pton(AF_INET, SERVER, &sa.sin_addr);

      err = connect(sd, (SA *) & sa, sizeof(sa));
      if (err < 0) {
         fprintf(stderr, "Connect error");
         exit(1);
      }
      gnutls_init(&state, GNUTLS_CLIENT);
      gnutls_set_protocol_priority(state, GNUTLS_TLS1, GNUTLS_SSL3, 0);
      gnutls_set_cipher_priority(state, GNUTLS_3DES_CBC, GNUTLS_ARCFOUR, 0);
      gnutls_set_compression_priority(state, GNUTLS_NULL_COMPRESSION, 0);
      gnutls_set_kx_priority(state, GNUTLS_KX_RSA, 0);
      gnutls_set_mac_priority(state, GNUTLS_MAC_SHA, GNUTLS_MAC_MD5, 0);


      gnutls_set_cred(state, GNUTLS_X509PKI, xcred);

      if (t > 0) { /* if this is not the first time we connect */
         gnutls_set_current_session(state, session, session_size);
         free(session);
      }
      /* Perform the TLS handshake
       */
      ret = gnutls_handshake(sd, state);

      if (ret < 0) {
         fprintf(stderr, "*** Handshake failed\n");
         gnutls_perror(ret);
         goto end;
      } else {
         printf("- Handshake was completed\n");
      }

      if (t == 0) { /* the first time we connect */
         /* get the session data size */
         gnutls_get_current_session(state, NULL, &session_size);
         session = malloc(session_size);

         /* put session data to the session variable */
         gnutls_get_current_session(state, session, &session_size);

         /* keep the current session ID. This is only needed
          * in order to check if the server actually resumed this
          * connection.
          */
         gnutls_get_current_session_id(state, NULL, &session_id_size);
         session_id = malloc(session_id_size);
         gnutls_get_current_session_id(state, session_id, &session_id_size);

      } else { /* the second time we connect */

         /* check if we actually resumed the previous session */
         gnutls_get_current_session_id(state, NULL, &tmp_session_id_size);
         tmp_session_id = malloc(tmp_session_id_size);
         gnutls_get_current_session_id(state, tmp_session_id, &tmp_session_id_size);

         if (memcmp(tmp_session_id, session_id, session_id_size) == 0) {
            printf("- Previous session was resumed\n");
         } else {
            fprintf(stderr, "*** Previous session was NOT resumed\n");
         }
         free(tmp_session_id);
         free(session_id);
      }

      /* This function was defined in a previous example
       */
      print_info(state);

      gnutls_write(sd, state, MSG, strlen(MSG));

      ret = gnutls_read(sd, state, buffer, MAX_BUF);
      if (gnutls_is_fatal_error(ret) == 1 || ret == 0) {
         if (ret == 0) {
            printf("- Peer has closed the GNUTLS connection\n");
            goto end;
         } else {
            fprintf(stderr, "*** Received corrupted data(%d) - server has terminated the connection abnormally\n",
                    ret);
            goto end;
         }
      } else {
         if (ret == GNUTLS_E_WARNING_ALERT_RECEIVED || ret == GNUTLS_E_FATAL_ALERT_RECEIVED)
            printf("* Received alert [%d]\n", gnutls_get_last_alert(state));
         if (ret == GNUTLS_E_GOT_HELLO_REQUEST)
            printf("* Received HelloRequest message (server asked to rehandshake)\n");
      }

      if (ret > 0) {
         printf("- Received %d bytes: ", ret);
         for (ii = 0; ii < ret; ii++) {
            fputc(buffer[ii], stdout);
         }
         fputs("\n", stdout);
      }
      gnutls_bye(sd, state, 0);

    end:

      shutdown(sd, SHUT_RDWR);  /* no more receptions */
      close(sd);

      gnutls_deinit(state);

   }  /* for() */

   gnutls_free_x509_client_sc(xcred);

   gnutls_global_deinit();

   return 0;
}

\end{verbatim}



\include{gnutls-api}

\end{document}

