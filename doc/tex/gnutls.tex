\documentclass{article}
\usepackage{html}
\begin{document}


\title{GNU Transport Layer Security Library}
\author{Nikos Mavroyanopoulos}
\maketitle

\tableofcontents

\section{Introduction}
\par
gnuTLS is a library which implements the {\bf TLS 1.0} and {\bf SSL 3.0} protocols.
TLS stands for 'Transport Layer Security' and is the sucessor of SSL (Secure Sockets Layer).
{\bf TLS 1.0} is described is {\it RFC 2246} and is an Internet protocol (thus it's mostly used over TCP/IP),
that provides confidentiality, and authentication layers. Currently {\bf gnuTLS} implements:
\begin{itemize}
 \item the {\bf TLS 1.0} and {\bf{ SSL 3.0}} protocols, without any (US) export-controlled algorithms
 \item {\bf X509} Public Key Infrastructure (with several limitations).
 \item {\bf SRP} for TLS authentication.
 \item {\bf TLS Extensions}
\end{itemize}

\subsection{Confidentiality}
\par
Confidentiality is provided by using symmetric encryption algorithms like {\bf 3DES}, {\bf AES}, or
stream algorithms like {\bf ARCFOUR}. A symmetric encryption algorithm uses a single (secret) key
to encrypt and decrypt data.

\subsection{Authentication}
\par
The following authentication schemas are supported in gnuTLS:
\begin{enumerate}
 \item X509 Public Key Infrastructure
 \item Anonymous authentication
 \item SRP authentication
\end{enumerate}



\section{Client Examples}
This section contains examples of TLS and SSL clients, using gnuTLS. 

\subsection{Simple Client example with X509 Authentication}
Let's assume now that we want to create a client which communicates
with servers using the X509 authentication schema. The following client
is a very simple TLS client, it does not support session resuming nor
any other fancy features.
\begin{verbatim}

#include <stdio.h>
#include <stdlib.h>
#include <sys/types.h>
#include <sys/socket.h>
#include <netinet/in.h>
#include <arpa/inet.h>
#include <gnutls.h>

#define MAX_BUF 1024
#define CRLFILE "crl.pem"
#define CAFILE "ca.pem"
#define SA struct sockaddr
#define MSG "GET / HTTP/1.0\r\n\r\n"

int main()
{
   const char *PORT = "443";
   const char *SERVER = "127.0.0.1";
   int err, ret;
   int sd, ii;
   struct sockaddr_in sa;
   GNUTLS_STATE state;
   char buffer[MAX_BUF + 1];
   X509PKI_CLIENT_CREDENTIALS xcred;

   if (gnutls_global_init() < 0) {
      fprintf(stderr, "global state initialization error\n");
      exit(1);
   }
   /* X509 stuff */
   if (gnutls_allocate_x509_client_sc(&xcred, 0) < 0) {  /* no client private key */
      fprintf(stderr, "memory error\n");
      exit(1);
   }
   /* set's the trusted cas file
    */
   gnutls_set_x509_client_trust(xcred, CAFILE, CRLFILE);

   /* connects to server 
    */
   sd = socket(AF_INET, SOCK_STREAM, 0);

   memset(&sa, '\0', sizeof(sa));
   sa.sin_family = AF_INET;
   sa.sin_port = htons(atoi(PORT));
   inet_pton(AF_INET, SERVER, &sa.sin_addr);

   err = connect(sd, (SA *) & sa, sizeof(sa));
   if (err < 0) {
      fprintf(stderr, "Connect error\n");
      exit(1);
   }
   /* Initialize TLS state 
    */
   gnutls_init(&state, GNUTLS_CLIENT);

   /* allow both SSL3 and TLS1
    */
   gnutls_set_protocol_priority(state, GNUTLS_TLS1, GNUTLS_SSL3, 0);

   /* allow only ARCFOUR and 3DES ciphers
    * (3DES has the highest priority)
    */
   gnutls_set_cipher_priority(state, GNUTLS_3DES_CBC, GNUTLS_ARCFOUR, 0);

   /* only allow null compression
    */
   gnutls_set_compression_priority(state, GNUTLS_NULL_COMPRESSION, 0);

   /* use GNUTLS_KX_RSA
    */
   gnutls_set_kx_priority(state, GNUTLS_KX_RSA, 0);

   /* allow the usage of both SHA and MD5
    */
   gnutls_set_mac_priority(state, GNUTLS_MAC_SHA, GNUTLS_MAC_MD5, 0);


   /* put the x509 credentials to the current state
    */
   gnutls_set_cred(state, GNUTLS_X509PKI, xcred);


   /* Perform the TLS handshake
    */
   ret = gnutls_handshake(sd, state);

   if (ret < 0) {
      fprintf(stderr, "*** Handshake failed\n");
      gnutls_perror(ret);
      goto end;
   } else {
      printf("- Handshake was completed\n");
   }

   gnutls_write(sd, state, MSG, strlen(MSG));

   ret = gnutls_read(sd, state, buffer, MAX_BUF);
   if (gnutls_is_fatal_error(ret) == 1 || ret == 0) {
      if (ret == 0) {
         printf("- Peer has closed the GNUTLS connection\n");
         goto end;
      } else {
         fprintf(stderr, "*** Received corrupted data(%d) - server has terminated the connection abnormally\n",
                 ret);
         goto end;
      }
   } else {
      if (ret == GNUTLS_E_WARNING_ALERT_RECEIVED || ret == GNUTLS_E_FATAL_ALERT_RECEIVED)
         printf("* Received alert [%d]\n", gnutls_get_last_alert(state));
      if (ret == GNUTLS_E_REHANDSHAKE)
         printf("* Received HelloRequest message (server asked to rehandshake)\n");
   }

   if (ret > 0) {
      printf("- Received %d bytes: ", ret);
      for (ii = 0; ii < ret; ii++) {
         fputc(buffer[ii], stdout);
      }
      fputs("\n", stdout);
   }
   gnutls_bye(sd, state, GNUTLS_SHUT_RDWR);

 end:

   shutdown(sd, SHUT_RDWR);     /* no more receptions */
   close(sd);

   gnutls_deinit(state);

   gnutls_free_x509_client_sc(xcred);

   gnutls_global_deinit();

   return 0;
}

\end{verbatim}


\subsection{Getting peer's information}
\par The above example was the simplest form of a client, it didn't even check
the result of the peer's certificate verification function (ie. if we have
an authenticated connection). The following function does check the peer's X509
Certificate, and prints some information about the current state.
\par
This function should be called after a successful gnutls\_handshake().
% \hyperref{gnutls\_handshake()}{gnutls\_handshake() (see Section }{ for more information)}{gnutls\_handshake}

\begin{verbatim}

#define PRINTX(x,y) if (y[0]!=0) printf(" -   %s %s\n", x, y)
#define PRINT_DN(X) PRINTX( "CN:", x509_info->X.common_name); \
	PRINTX( "OU:", x509_info->X.organizational_unit_name); \
	PRINTX( "O:", x509_info->X.organization); \
	PRINTX( "L:", x509_info->X.locality_name); \
	PRINTX( "S:", x509_info->X.state_or_province_name); \
	PRINTX( "C:", x509_info->X.country);

int print_info(GNUTLS_STATE state)
{
   const char *tmp;
   const X509PKI_CLIENT_AUTH_INFO *x509_info;

   /* print the key exchange's algorithm name
    */
   tmp = gnutls_kx_get_name(gnutls_get_current_kx(state));
   printf("- Key Exchange: %s\n", tmp);

   /* in case of X509 PKI
    */
   if (gnutls_get_auth_info_type(state) == GNUTLS_X509PKI) {
      x509_info = gnutls_get_auth_info(state);
      if (x509_info != NULL) {
         switch (x509_info->peer_certificate_status) {
         case GNUTLS_CERT_NOT_TRUSTED:
            printf("- Peer's X509 Certificate was NOT verified\n");
            break;
         case GNUTLS_CERT_EXPIRED:
            printf("- Peer's X509 Certificate was verified but is expired\n");
            break;
         case GNUTLS_CERT_TRUSTED:
            printf("- Peer's X509 Certificate was verified\n");
            break;
         case GNUTLS_CERT_INVALID:
         default:
            printf("- Peer's X509 Certificate was invalid\n");
            break;

         }
      }
   }
   printf(" - Certificate info:\n");
   printf(" - Certificate version: #%d\n", x509_info->peer_certificate_version);

   PRINT_DN(peer_dn);

   printf(" - Certificate Issuer's info:\n");
   PRINT_DN(issuer_dn);


   tmp = gnutls_version_get_name(gnutls_get_current_version(state));
   printf("- Version: %s\n", tmp);

   tmp = gnutls_compression_get_name(gnutls_get_current_compression_method(state));
   printf("- Compression: %s\n", tmp);

   tmp = gnutls_cipher_get_name(gnutls_get_current_cipher(state));
   printf("- Cipher: %s\n", tmp);

   tmp = gnutls_mac_get_name(gnutls_get_current_mac_algorithm(state));
   printf("- MAC: %s\n", tmp);

   return 0;
}

\end{verbatim}


\subsection{Resuming Sessions}
\par
The gnutls\_handshake() function, is expensive since
a lot of calculations are performed. In order to support many fast connections to
the same server a client may use session resuming. {\bf Session resuming} is a
feature of the {\bf TLS} protocol which allows a client to connect to a server,
after a successful handshake, without the expensive calculations (ie. use the previously
established keys). {\bf gnuTLS} supports this feature, and this example illustrates a
typical use of it (This is a modification of the simple client example).

\par
Keep in mind that sessions are expired after some time (for security reasons), thus
it may be normal for a server not to resume a session even if you requested that.

\begin{verbatim}

#include <stdio.h>
#include <stdlib.h>
#include <sys/types.h>
#include <sys/socket.h>
#include <netinet/in.h>
#include <arpa/inet.h>
#include <gnutls.h>

#define MAX_BUF 1024
#define CRLFILE "crl.pem"
#define CAFILE "ca.pem"
#define SA struct sockaddr
#define MSG "GET / HTTP/1.0\r\n\r\n"

int main()
{
   const char *PORT = "443";
   const char *SERVER = "127.0.0.1";
   int err, ret;
   int sd, ii;
   struct sockaddr_in sa;
   GNUTLS_STATE state;
   char buffer[MAX_BUF + 1];
   X509PKI_CLIENT_CREDENTIALS xcred;
   /* variables used in session resuming */
   int t;
   char *session;
   char *session_id;
   int session_size;
   int session_id_size;
   char *tmp_session_id;
   int tmp_session_id_size;

   if (gnutls_global_init() < 0) {
      fprintf(stderr, "global state initialization error\n");
      exit(1);
   }
   /* X509 stuff */
   if (gnutls_allocate_x509_client_sc(&xcred, 0) < 0) {  /* no client private key */
      fprintf(stderr, "memory error\n");
      exit(1);
   }
   gnutls_set_x509_client_trust(xcred, CAFILE, CRLFILE);

   for (t = 0; t < 2; t++) {    /* connect 2 times to the server */

      sd = socket(AF_INET, SOCK_STREAM, 0);
      memset(&sa, '\0', sizeof(sa));
      sa.sin_family = AF_INET;
      sa.sin_port = htons(atoi(PORT));
      inet_pton(AF_INET, SERVER, &sa.sin_addr);

      err = connect(sd, (SA *) & sa, sizeof(sa));
      if (err < 0) {
         fprintf(stderr, "Connect error");
         exit(1);
      }
      gnutls_init(&state, GNUTLS_CLIENT);
      gnutls_set_protocol_priority(state, GNUTLS_TLS1, GNUTLS_SSL3, 0);
      gnutls_set_cipher_priority(state, GNUTLS_3DES_CBC, GNUTLS_ARCFOUR, 0);
      gnutls_set_compression_priority(state, GNUTLS_NULL_COMPRESSION, 0);
      gnutls_set_kx_priority(state, GNUTLS_KX_RSA, 0);
      gnutls_set_mac_priority(state, GNUTLS_MAC_SHA, GNUTLS_MAC_MD5, 0);


      gnutls_set_cred(state, GNUTLS_X509PKI, xcred);

      if (t > 0) { /* if this is not the first time we connect */
         gnutls_set_current_session(state, session, session_size);
         free(session);
      }
      /* Perform the TLS handshake
       */
      ret = gnutls_handshake(sd, state);

      if (ret < 0) {
         fprintf(stderr, "*** Handshake failed\n");
         gnutls_perror(ret);
         goto end;
      } else {
         printf("- Handshake was completed\n");
      }

      if (t == 0) { /* the first time we connect */
         /* get the session data size */
         gnutls_get_current_session(state, NULL, &session_size);
         session = malloc(session_size);

         /* put session data to the session variable */
         gnutls_get_current_session(state, session, &session_size);

         /* keep the current session ID. This is only needed
          * in order to check if the server actually resumed this
          * connection.
          */
         gnutls_get_current_session_id(state, NULL, &session_id_size);
         session_id = malloc(session_id_size);
         gnutls_get_current_session_id(state, session_id, &session_id_size);

      } else { /* the second time we connect */

         /* check if we actually resumed the previous session */
         gnutls_get_current_session_id(state, NULL, &tmp_session_id_size);
         tmp_session_id = malloc(tmp_session_id_size);
         gnutls_get_current_session_id(state, tmp_session_id, &tmp_session_id_size);

         if (memcmp(tmp_session_id, session_id, session_id_size) == 0) {
            printf("- Previous session was resumed\n");
         } else {
            fprintf(stderr, "*** Previous session was NOT resumed\n");
         }
         free(tmp_session_id);
         free(session_id);
      }

      /* This function was defined in a previous example
       */
      print_info(state);

      gnutls_write(sd, state, MSG, strlen(MSG));

      ret = gnutls_read(sd, state, buffer, MAX_BUF);
      if (gnutls_is_fatal_error(ret) == 1 || ret == 0) {
         if (ret == 0) {
            printf("- Peer has closed the GNUTLS connection\n");
            goto end;
         } else {
            fprintf(stderr, "*** Received corrupted data(%d) - server has terminated the connection abnormally\n",
                    ret);
            goto end;
         }
      } else {
         if (ret == GNUTLS_E_WARNING_ALERT_RECEIVED || ret == GNUTLS_E_FATAL_ALERT_RECEIVED)
            printf("* Received alert [%d]\n", gnutls_get_last_alert(state));
         if (ret == GNUTLS_E_GOT_HELLO_REQUEST)
            printf("* Received HelloRequest message (server asked to rehandshake)\n");
      }

      if (ret > 0) {
         printf("- Received %d bytes: ", ret);
         for (ii = 0; ii < ret; ii++) {
            fputc(buffer[ii], stdout);
         }
         fputs("\n", stdout);
      }
      gnutls_bye(sd, state, 0);

    end:

      shutdown(sd, SHUT_RDWR);  /* no more receptions */
      close(sd);

      gnutls_deinit(state);

   }  /* for() */

   gnutls_free_x509_client_sc(xcred);

   gnutls_global_deinit();

   return 0;
}

\end{verbatim}


\subsection{Simple Client example with SRP Authentication}
Although {\bf SRP} is not part of the TLS standard, gnuTLS implements
{\it draft-ietf-tls-srp-01} which defines a way to use the SRP algorithm
within the TLS handshake. The following client
is a very simple SRP-TLS client which connects to a server by using 
{\it username} and {\it password}.

\begin{verbatim}

#include <stdio.h>
#include <stdlib.h>
#include <sys/types.h>
#include <sys/socket.h>
#include <netinet/in.h>
#include <arpa/inet.h>
#include <gnutls.h>

#define MAX_BUF 1024
#define USERNAME "user"
#define PASSWORD "pass"
#define SA struct sockaddr
#define MSG "GET / HTTP/1.0\r\n\r\n"

const int protocol_priority[] = { GNUTLS_TLS1, GNUTLS_SSL3, 0 };
const int kx_priority[] = { GNUTLS_KX_SRP, 0 };
const int cipher_priority[] = { GNUTLS_CIPHER_3DES_CBC, GNUTLS_CIPHER_ARCFOUR, 0};
const int comp_priority[] = { GNUTLS_COMP_NULL, 0 };
const int mac_priority[] = { GNUTLS_MAC_SHA, GNUTLS_MAC_MD5, 0 };

int main()
{
   const char *PORT = "443";
   const char *SERVER = "127.0.0.1";
   int err, ret;
   int sd, ii;
   struct sockaddr_in sa;
   GNUTLS_STATE state;
   char buffer[MAX_BUF + 1];
   SRP_CLIENT_CREDENTIALS xcred;

   if (gnutls_global_init() < 0) {
      fprintf(stderr, "global state initialization error\n");
      exit(1);
   }
   if (gnutls_srp_allocate_client_sc(&xcred) < 0) {
      fprintf(stderr, "memory error\n");
      exit(1);
   }
   gnutls_srp_set_client_cred(xcred, USERNAME, PASSWORD);

   /* connects to server 
    */
   sd = socket(AF_INET, SOCK_STREAM, 0);

   memset(&sa, '\0', sizeof(sa));
   sa.sin_family = AF_INET;
   sa.sin_port = htons(atoi(PORT));
   inet_pton(AF_INET, SERVER, &sa.sin_addr);

   err = connect(sd, (SA *) & sa, sizeof(sa));
   if (err < 0) {
      fprintf(stderr, "Connect error\n");
      exit(1);
   }
   /* Initialize TLS state 
    */
   gnutls_init(&state, GNUTLS_CLIENT);

   /* allow both SSL3 and TLS1
    */
   gnutls_protocol_set_priority(state, protocol_priority);
 
   /* allow only ARCFOUR and 3DES ciphers
    * (3DES has the highest priority)
    */
    gnutls_cipher_set_priority(state, cipher_priority);

   /* only allow null compression
    */
   gnutls_compression_set_priority(state, comp_priority);
 
   /* use GNUTLS_KX_RSA
    */
   gnutls_kx_set_priority(state, kx_priority);
 
   /* allow the usage of both SHA and MD5
    */
   gnutls_mac_set_priority(state, mac_priority);


   /* put the SRP credentials to the current state
    */
   gnutls_set_cred(state, GNUTLS_SRP, xcred);

   gnutls_transport_set_ptr( state, sd);

   /* Perform the TLS handshake
    */
   ret = gnutls_handshake( state);

   if (ret < 0) {
      fprintf(stderr, "*** Handshake failed\n");
      gnutls_perror(ret);
      goto end;
   } else {
      printf("- Handshake was completed\n");
   }

   gnutls_write( state, MSG, strlen(MSG));

   ret = gnutls_read( state, buffer, MAX_BUF);
   if (gnutls_is_fatal_error(ret) == 1 || ret == 0) {
      if (ret == 0) {
         printf("- Peer has closed the GNUTLS connection\n");
         goto end;
      } else {
         fprintf(stderr, "*** Received corrupted data(%d) - server has terminated the connection abnormally\n",
                 ret);
         goto end;
      }
   } else {
      if (ret == GNUTLS_E_WARNING_ALERT_RECEIVED || ret == GNUTLS_E_FATAL_ALERT_RECEIVED)
         printf("* Received alert [%d]\n", gnutls_get_last_alert(state));
      if (ret == GNUTLS_E_REHANDSHAKE)
         printf("* Received HelloRequest message (server asked to rehandshake)\n");
   }

   if (ret > 0) {
      printf("- Received %d bytes: ", ret);
      for (ii = 0; ii < ret; ii++) {
         fputc(buffer[ii], stdout);
      }
      fputs("\n", stdout);
   }
   gnutls_bye( state, 0);

 end:

   shutdown(sd, SHUT_RDWR);     /* no more receptions */
   close(sd);

   gnutls_deinit(state);

   gnutls_srp_free_client_sc(xcred);

   gnutls_global_deinit();

   return 0;
}

\end{verbatim}


\section{Server Examples}
This section contains examples of TLS and SSL servers, using gnuTLS.

\subsection{Echo Server with X509 and SRP authentication}
The following example is a server which supports both {\bf SRP} and {\bf X509} authentication.
This server also supports {\it session resuming}.
\begin{verbatim}

#include <stdio.h>
#include <stdlib.h>
#include <errno.h>
#include <sys/types.h>
#include <sys/socket.h>
#include <netinet/in.h>
#include <arpa/inet.h>
#include <string.h>
#include <unistd.h>
#include <gnutls.h>

#define KEYFILE "key.pem"
#define CERTFILE "cert.pem"
#define CAFILE "ca.pem"
#define CRLFILE NULL

#define SRP_PASSWD "tpasswd"
#define SRP_PASSWD_CONF "tpasswd.conf"


/* This is a sample TCP echo server.
 */


#define SA struct sockaddr
#define ERR(err,s) if(err==-1) {perror(s);return(1);}
#define MAX_BUF 1024
#define PORT 5556               /* listen to 5556 port */

/* These are global */
GNUTLS_SRP_SERVER_CREDENTIALS srp_cred;
GNUTLS_X509PKI_SERVER_CREDENTIALS x509_cred;

GNUTLS_STATE initialize_state()
{
   GNUTLS_STATE state;
   int ret;
   const int protocol_priority[] = { GNUTLS_TLS1, GNUTLS_SSL3, 0 };
   const int kx_priority[] = { GNUTLS_KX_RSA, GNUTLS_KX_DHE_RSA, GNUTLS_KX_SRP, 0 };
   const int cipher_priority[] = { GNUTLS_CIPHER_RIJNDAEL_CBC, GNUTLS_CIPHER_3DES_CBC, 0};
   const int comp_priority[] = { GNUTLS_COMP_ZLIB, GNUTLS_COMP_NULL, 0 };
   const int mac_priority[] = { GNUTLS_MAC_SHA, GNUTLS_MAC_MD5, 0 };

   gnutls_init(&state, GNUTLS_SERVER);

   /* in order to support session resuming:
    */
   if ((ret = gnutls_db_set_name(state, "gnutls-rsm.db")) < 0)
      fprintf(stderr, "*** DB error (%d)\n\n", ret);

   gnutls_protocol_set_priority(state, protocol_priority);
   gnutls_cipher_set_priority(state, cipher_priority);
   gnutls_compression_set_priority(state, comp_priority);
   gnutls_kx_set_priority(state, kx_priority);
   gnutls_mac_set_priority(state, mac_priority);

   gnutls_cred_set(state, GNUTLS_SRP, srp_cred);
   gnutls_cred_set(state, GNUTLS_X509PKI, x509_cred);

   /* request client certificate if any.
    */
   gnutls_x509pki_server_set_cert_request( state, GNUTLS_CERT_REQUEST);
   
   return state;
}

void print_info(GNUTLS_STATE state)
{
   const char *tmp;
   unsigned char sesid[32];
   int sesid_size, i;

   /* print session_id specific data */
   gnutls_session_get_id(state, sesid, &sesid_size);
   printf("\n- Session ID: ");
   for (i = 0; i < sesid_size; i++)
      printf("%.2X", sesid[i]);
   printf("\n");

   /* print srp specific data */
   if (gnutls_get_auth_type(state) == GNUTLS_SRP) {
         printf("\n- User '%s' connected\n",
                gnutls_srp_server_get_username( state));
   }

   /* print state information */
   tmp = gnutls_protocol_get_name(gnutls_protocol_get_version(state));
   printf("- Version: %s\n", tmp);

   tmp = gnutls_kx_get_name(gnutls_kx_get(state));
   printf("- Key Exchange: %s\n", tmp);

   tmp =
       gnutls_compression_get_name
       (gnutls_compression_get(state));
   printf("- Compression: %s\n", tmp);

   tmp = gnutls_cipher_get_name(gnutls_cipher_get(state));
   printf("- Cipher: %s\n", tmp);

   tmp = gnutls_mac_get_name(gnutls_mac_get(state));
   printf("- MAC: %s\n", tmp);

}



int main()
{
   int err, listen_sd, i;
   int sd, ret;
   struct sockaddr_in sa_serv;
   struct sockaddr_in sa_cli;
   int client_len;
   char topbuf[512];
   GNUTLS_STATE state;
   char buffer[MAX_BUF + 1];
   int optval = 1;
   int http = 0;
   char name[256];

   strcpy(name, "Echo Server");

   /* this must be called once in the program
    */
   if (gnutls_global_init() < 0) {
      fprintf(stderr, "global state initialization error\n");
      exit(1);
   }
   if (gnutls_x509pki_allocate_server_sc(&x509_cred, 1) < 0) {
      fprintf(stderr, "memory error\n");
      exit(1);
   }
   if (gnutls_x509pki_set_server_trust_file(x509_cred, CAFILE, CRLFILE) < 0) {
      fprintf(stderr, "X509 PARSE ERROR\nDid you have ca.pem?\n");
      exit(1);
   }
   if (gnutls_x509pki_set_server_key_file(x509_cred, CERTFILE, KEYFILE) < 0) {
      fprintf(stderr, "X509 PARSE ERROR\nDid you have key.pem and cert.pem?\n");
      exit(1);
   }
   /* SRP_PASSWD a password file (created with the included crypt utility) 
    * Read README.crypt prior to using SRP.
    */
   gnutls_srp_allocateserver_sc(&srp_cred);
   gnutls_srp_set_server_cred(srp_cred, SRP_PASSWD, SRP_PASSWD_CONF);


   /* Socket operations
    */
   listen_sd = socket(AF_INET, SOCK_STREAM, 0);
   ERR(listen_sd, "socket");

   memset(&sa_serv, '\0', sizeof(sa_serv));
   sa_serv.sin_family = AF_INET;
   sa_serv.sin_addr.s_addr = INADDR_ANY;
   sa_serv.sin_port = htons(PORT);  /* Server Port number */

   setsockopt(listen_sd, SOL_SOCKET, SO_REUSEADDR, &optval, sizeof(int));

   err = bind(listen_sd, (SA *) & sa_serv, sizeof(sa_serv));
   ERR(err, "bind");
   err = listen(listen_sd, 1024);
   ERR(err, "listen");

   printf("%s ready. Listening to port '%d'.\n\n", name, PORT);

   client_len = sizeof(sa_cli);
   for (;;) {
      state = initialize_state();

      sd = accept(listen_sd, (SA *) & sa_cli, &client_len);

      printf("- connection from %s, port %d\n",
             inet_ntop(AF_INET, &sa_cli.sin_addr, topbuf,
                       sizeof(topbuf)), ntohs(sa_cli.sin_port));

      gnutls_transport_set_ptr( state, sd);
      ret = gnutls_handshake( state);
      if (ret < 0) {
         close(sd);
         gnutls_deinit(state);
         fprintf(stderr, "*** Handshake has failed (%s)\n\n",
                 gnutls_strerror(ret));
         continue;
      }
      printf("- Handshake was completed\n");

      print_info(state);

      i = 0;
      for (;;) {
         bzero(buffer, MAX_BUF + 1);
         ret = gnutls_read( state, buffer, MAX_BUF);

         if (gnutls_error_is_fatal(ret) == 1 || ret == 0) {
            if (ret == 0) {
               printf
                   ("\n- Peer has closed the GNUTLS connection\n");
               break;
            } else {
               fprintf(stderr,
                       "\n*** Received corrupted data(%d). Closing the connection.\n\n",
                       ret);
               break;
            }

         }
         if (ret > 0) {
            /* echo data back to the client
             */
            gnutls_write( state, buffer,
                         strlen(buffer));
         }
         if (ret == GNUTLS_E_WARNING_ALERT_RECEIVED || ret == GNUTLS_E_FATAL_ALERT_RECEIVED) {
            ret = gnutls_alert_get_last(state);
            printf("* Received alert '%d'.\n", ret);
         }
      }
      printf("\n");
      gnutls_bye( state, 1); /* do not wait for
                                 * the peer to close the connection.
                                 */

      close(sd);
      gnutls_deinit(state);

   }
   close(listen_sd);

   gnutls_x509pki_free_server_sc(x509_cred);
   gnutls_srp_free_server_sc(srp_cred);

   gnutls_global_deinit();

   return 0;

}

\end{verbatim}


\include{gnutls-api}

\end{document}

