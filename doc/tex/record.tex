\section{The TLS record protocol\index{TLS protocols!Record}}

The Record protocol is the secure communications provider. It's purpose
is to encrypt, authenticate and --optionally-- compress packets.
The following functions are available:
\par
\begin{itemize}
\item \printfunc{gnutls_record_send}{gnutls\_record\_send}:
to send a record packet (with application data).
\item \printfunc{gnutls_record_recv}{gnutls\_record\_recv}:
to receive a record packet (with application data).
\end{itemize}

As you may have already noticed, the functions which access the Record protocol,
are quite limited, given the importance of this protocol in \tls{}.
This is because the Record protocol's parameters are all set by
the Handshake protocol (see section \ref{handshake} on page \pageref{handshake}).
\par
The Record protocol initially starts with NULL parameters, which means
no encryption, and no MAC is used. Encryption and authentication begin
just after the handshake protocol has finished.

\subsection{Encryption algorithms used in the record layer}
\index{Symmetric encryption algorithms}
Confidentiality in the record layer is achieved by using symmetric block 
encryption algorithms like {\bf 3DES}, {\bf AES\footnote{AES or Advanced 
Encryption Standard is actually the RIJNDAEL algorithm. This is the
algorithm that replaced DES.}}, or
stream algorithms like {\bf ARCFOUR\_128\footnote{ARCFOUR\_128 is a compatible
algorithm with RSA's RC4 algorithm, which is considered to be a trade secret.}} See \hyperref{fig:ciphers}{figure }{}{fig:ciphers} for a complete list. 
Ciphers are encryption algorithms that use a single (secret) key
to encrypt and decrypt data. Block algorithms in TLS also provide protection
against statistical analysis of the data. \gnutls{} makes use of this property
thus, if you're using the \tlsI{} protocol, a random number of blocks will be
appended to the data. This will prevent eavesdroppers from guessing the 
actual data size.

\begin{figure}[hbtp]
\begin{tabular}{|l|p{9cm}|}

\hline
3DES\_CBC & 3DES\_CBC is the DES block cipher algorithm used with triple
encryption (EDE). Has 64 bits block size and is used in CBC mode.
\\
\hline
ARCFOUR\_128 & ARCFOUR is a fast stream cipher.
\\
\hline
ARCFOUR\_40 & This is the ARCFOUR cipher that is fed with a 40 bit key,
which is considered weak.
\\
\hline
AES\_CBC & AES or RIJNDAEL is the block cipher algorithm that replaces 
the old DES algorithm. Has
128 bits block size and is used in CBC mode. This is not officially
supported in TLS.
\\
\hline
\end{tabular}
\caption{Supported cipher algorithms}
\label{fig:ciphers}
\end{figure}



\addvspace{1.5cm}

\begin{figure}[hbtp]
\begin{tabular}{|l|p{9cm}|}

\hline
MAC\_MD5 & MD5 is a cryptographic hash algorithm designed by Ron Rivest. Outputs 128 bits of data.
\\
\hline
MAC\_SHA & SHA is a cryptographic hash algorithm designed by NSA. Outputs 160 bits of data.
\\
\hline
MAC\_RMD160 & RIPEMD is a cryptographic hash algorithm developed in the framework
of the EU project RIPE. Outputs 160 bits of data.
\\
\hline
\end{tabular}
\caption{Supported MAC algorithms}
\index{MAC algorithms}
\label{fig:mac}
\end{figure}



\subsection{Compression algorithms used in the record layer}
\index{Compression algorithms}
The TLS' record layer also supports compression. The algorithms
implemented in \gnutls{} can found in figure \ref{fig:compression}.
All the algorithms should be considered as \gnutls' extensions, and
should be advertised only when the peer is known to have a compliant client,
to avoid interoperability problems.
\par
The included algorithms perform really good when text, or other
compressable data are to be transfered, but offer nothing on already 
compressed data, such as compressed images, zipped archives etc.
These compression algorithms, may be useful in high bandwidth TLS tunnels,
and in cases where network usage has to be minimized. As a drawback, 
compression increases latency.

\begin{figure}[hbtp]
\begin{tabular}{|l|p{9cm}|}

\hline
ZLIB & ZLIB compression, using the deflate algorithm.
\\
\hline
LZO & LZO is a very fast compression algorithm. This algorithm is only
available if the \gnutlse{} library has been initialized.
\\
\hline
\end{tabular}
\caption{Supported compression algorithms}
\label{fig:compression}
\end{figure}




\subsection{Weaknesses and countermeasures}
\index{TLS protocols!Record}

Some weaknesses that may affect the security of the Record layer have been
found in \tlsI{} protocol. These weaknesses can be exploited by active attackers,
and exploit the facts that \tls{} 
\begin{enumerate}
\item has separate alerts for ``decryption\_failed'' and ``bad\_record\_mac''
\item the decryption failure reason can be detected by timing the responce time
\item the IV for CBC encrypted packets is the last block of the previous encrypted packet
\end{enumerate}

\gnutls{} implements all the known counter-measures for these attacks. For the first
two cases, \gnutls{} does only have one error code for both of the decryption failures,
and processes the message normaly even if a padding error occured. This avoids
both of these attacks.
For the latter, an empty record can be sent before every record packet, and this is
believed to avoid the known attacks in CBC encrypted packets. See the function
\printfunc{gnutls_record_set_cbc_protection}{gnutls\_record\_set\_cbc\_protection}
for more information.

For a detailed discussion see the archives of the TLS Working Group mailing list
and the paper \cite{CBCATT}.



