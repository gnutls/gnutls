\section{The TLS record protocol\index{Record protocol}}

The Record protocol is the secure communications provider. It's job is
to encrypt and authenticate packets. 
The following functions are available:
\par
\begin{itemize}
\item \printfunc{gnutls_record_send}{gnutls\_record\_send}:
to send an record packet (with application data).
\item \printfunc{gnutls_record_recv}{gnutls\_record\_recv}:
to receive a record packet (with application data).
\end{itemize}

As you may have already noticed, the functions which access the Record protocol,
are quite limited, given the importance of this protocol in \tls{}.
This is because the Record protocol's parameters are all set by
the Handshake protocol (see section \ref{handshake} on page \pageref{handshake}).
\par
The Record protocol initialy starts with NULL parameters, which means
no encryption, and no MAC is used. Encryption and authentication begin
just after the handshake protocol has finished.

\subsection{Encryption algorithms used in the record layer}
\index{Symmetric encryption algorithms}
Confidentiality in the record layer is achieved by using symmetric block 
encryption algorithms like {\bf 3DES}, {\bf AES\footnote{AES or Advanced 
Encryption Standard is actually the RIJNDAEL algorithm. This is the
algorithm that replaced DES.}}, or
stream algorithms like {\bf ARCFOUR\_128\footnote{ARCFOUR\_128 is a compatible
algorithm with RSA's RC4 algorithm, which is considered to be a trade secret.}} See \hyperref{fig:ciphers}{figure }{}{fig:ciphers} for a complete list. 
Ciphers are encryption algorithms that use a single (secret) key
to encrypt and decrypt data. Block algorithms in TLS also provide protection
against statistical analysis of the data. \gnutls{} makes use of this property
thus, if you're using the \tlsI{} protocol, a random number of blocks will be
appended to the data. This will prevent eavesdroppers from guessing the 
actual data size.

\begin{figure}[hbtp]
\begin{tabular}{|l|p{9cm}|}

\hline
3DES\_CBC & 3DES\_CBC is the DES block cipher algorithm used with triple
encryption (EDE). Has 64 bits block size and is used in CBC mode.
\\
\hline
ARCFOUR\_128 & ARCFOUR is a fast stream cipher.
\\
\hline
ARCFOUR\_40 & This is the ARCFOUR cipher that is fed with a 40 bit key,
which is considered weak.
\\
\hline
AES\_CBC & AES or RIJNDAEL is the block cipher algorithm that replaces 
the old DES algorithm. Has
128 bits block size and is used in CBC mode. This is not officially
supported in TLS.
\\
\hline
\end{tabular}
\caption{Supported cipher algorithms}
\label{fig:ciphers}
\end{figure}



\addvspace{1.5cm}

\begin{figure}[hbtp]
\begin{tabular}{|l|p{9cm}|}

\hline
MAC\_MD5 & MD5 is a cryptographic hash algorithm designed by Ron Rivest. Outputs 128 bits of data.
\\
\hline
MAC\_SHA & SHA is a cryptographic hash algorithm designed by NSA. Outputs 160 bits of data.
\\
\hline
MAC\_RMD160 & RIPEMD is a cryptographic hash algorithm developed in the framework
of the EU project RIPE. Outputs 160 bits of data.
\\
\hline
\end{tabular}
\caption{Supported MAC algorithms}
\index{MAC algorithms}
\label{fig:mac}
\end{figure}



